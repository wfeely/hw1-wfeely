% hw1-wfeely-report.tex
\documentclass[11pt]{article}

\begin{document}
\title{\textbf{Design and Engineering of Intelligent Information Systems (DEIIS)}\\
	Homework \#1}
\author{Weston Feely \\ andrew ID: wfeely}
\date{September 11, 2013}
\maketitle

\section*{Type System Description}
My type system includes the following types.

\subsection{BaseAnnotation}
The base annotation type required by the assignment description. This includes two features. \\ \\
Features:
\begin{itemize}
\item string source: string to help keep track of where an annotation was originally made.
\item double confidence: how confident the annotation was.
\end{itemize}

\subsection{Question}
The question type for representing questions from the input data. This includes six features. \\ \\
Features:
\begin{itemize}
\item int begin: the beginning character offset for the question string.
\item int end: the ending character offset for the question string.
\item FSArray(Token) tokens: a Token array of tokens, for representing words from the question string.
\item FSArray(Ngram) unigrams: a Ngram array of bigrams, for representing 1-token strings from the question 
string. (similar to tokens).
\item FSArray(Ngram) bigrams: a Ngram array of bigrams, for representing 2-token strings from the question 
string.
\item FSArray(Ngram) trigrams: a Ngram array of trigrams, for representing 3-token strings from the question 
string.
\end{itemize}

\subsection{Answer}
The answer type for representing answers from the input data. This includes eight features. \\ \\
Features:
\begin{itemize}
\item int begin: the beginning character offset for the answer string.
\item int end: the ending character offset for the answer string.
\item FSArray(Token) tokens: a Token array of tokens, for representing words from the answer string.
\item FSArray(Ngram) unigrams: a Ngram array of bigrams, for representing 1-token strings from the answer 
string. (similar to tokens).
\item FSArray(Ngram) bigrams: a Ngram array of bigrams, for representing 2-token strings from the answer 
string.
\item FSArray(Ngram) trigrams: a Ngram array of trigrams, for representing 3-token strings from the answer 
string.
\item boolean gold: boolean for representing the 1 or 0 from the answer input text, which represents whether 
this is a good answer or a bad answer.
\item boolean guess: boolean for representing the guess (0 or 1) for whether this answer is good (1) or bad 
(0), based on the other features (excluding the gold variable).
\end{itemize}

\subsection{Token}
Token type for representing the words in a sentence. This includes three features. \\ \\
Features:
\begin{itemize}
\item int begin: the beginning character offset for the token.
\item int end: the ending character offset for the token.
\item string token: the string for the word token, itself.
\end{itemize}

\subsection{Ngram}
Ngram type for representing the ngrams (space-separated strings of words) in a sentence. This includes four 
features. \\ \\
Features:
\begin{itemize}
\item int order: the order of the ngram represented (1 for unigram, 2 for bigram, 3 for trigram).
\item int begin: the beginning character offset for the ngram.
\item int end: the ending character offset for the ngram.
\item string ngram: the string for the ngram, itself.
\end{itemize}

\subsection{Eval}
Evaluation type for representing evaluation objects. These objects wrap together the answers, and their 
corresponding question, along with the features necessary to perform scoring and ranking of answers, as well
as the final performance evaluation for a given input data set. This includes five features. \\ \\
Features:
\begin{itemize}
\item Question question: question object for the question in this data set.
\item FSArray(Answer) answers: answer object array for the answers in this data set.
\item doubleArray performance: scores for each answer in this data set.
\item int n: the number of correct answers in this data set.
\item double p: the precision for this data set (\# answers guessed correct divided by the number of total 
correct answers in this data set, n).
\end{itemize}

\end{document}
